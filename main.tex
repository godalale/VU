\documentclass[a4paper, 12pt, oneside]{book}

%\usepackage[T1]{fontenc}
\usepackage[utf8]{inputenc}

\usepackage[english]{babel}
\usepackage{geometry}
\usepackage{hhline}
\usepackage{amsthm}
\usepackage{amsmath}
\usepackage{amssymb}
\usepackage{comment}
\usepackage{enumerate}
\usepackage{mathrsfs}
\usepackage{multirow}
\usepackage{fullpage}
\usepackage{indentfirst}
\usepackage{color}
\usepackage{courier}
\usepackage{afterpage}
\usepackage{url}
\usepackage{float}
\usepackage{tabularx}
\usepackage[hidelinks]{hyperref}
\usepackage{cleveref}
\usepackage{epsfig}
\usepackage{pdfpages}
\usepackage{lastpage}
\usepackage{caption}
\usepackage{subcaption}
\usepackage{tikz}
\usepackage{siunitx}
\usepackage{listings}
\usepackage{lipsum}

\usepackage{setspace}
\usepackage{fancyhdr}

\renewcommand{\chaptermark}[1]{%
\markboth{\MakeUppercase{%
\chaptername}\ \thechapter.%
\ #1}{}}
\pagestyle{fancy}
\fancyhf{}
\fancyhead[L]{\leftmark}
\fancyhead[R]{\thepage}

\setlength{\headheight}{15pt}
\setlength{\headsep}{20pt}

\fancypagestyle{plain}{
  \fancyhf{} 
  \renewcommand{\headrulewidth}{0pt}
  \renewcommand{\footrulewidth}{0pt}
  \renewcommand{\headheight}{0pt}
  \renewcommand{\headsep}{0pt}
}

\usetikzlibrary{matrix,decorations.pathmorphing}

\graphicspath{/Users/alexandergodal/Desktop/BP/latex/img} 

\hypersetup{
    colorlinks=false,
    linkcolor=red,
    filecolor=magenta,      
    urlcolor=cyan,
    citecolor=blue
}

\usepackage{booktabs}

%%%%%%%%%%%%%%%%%%%%%%%%%%%%%%%%%%%%%%%%%
%%%% General macros %%%%%%%%%%%%%%%%%%%%%
%%%%%%%%%%%%%%%%%%%%%%%%%%%%%%%%%%%%%%%%%

\newcommand{\cvuten}{Czech Technical University in~Prague}
\newcommand{\fjfien}{Faculty of Nuclear Sciences and Physical Engineering}
\newcommand{\kfen}{Department of Physics}
\newcommand{\oboren}{Nuclear and Particle Physics}

\newcommand{\cvut}{České vysoké učení technické v Praze}
\newcommand{\fjfi}{Fakulta jaderná a fyzikálně inženýrská}
\newcommand{\kf}{Katedra fyziky}
\newcommand{\obor}{Jaderná a částicová fyzika}

\newcommand{\nazevcz}{Simulace hadronového kalorimetru nHCal pro ePIC experiment}
\newcommand{\klicovaslovacz}{EIC, experiment ePIC, hadronový kalorimetr}
\newcommand{\abstraktcz}{}

\newcommand{\nazeven}{Simulation of hadronic calorimeter nHCal for ePIC experiment} %skontrolovať názov – Hadron(-ic)
\newcommand{\klicovaslovaen}{EIC, ePIC Experiment, Hadronic Calorimeter}
\newcommand{\abstrakten}{}



\newcommand{\autor}{Bc. Alexander Godál}           
\newcommand{\rok}{2025}                 
\newcommand{\vedouci}{doc.~Mgr.~Jaroslav~Bielčík,~Ph.D.}
\newcommand{\konzultant}{Ing.~Alexandr~Prozorov,~Ph.D.}  

\newcommand{\akrok}{2024/2025}

\newcommand{\prohlaseni}{}

                                                               
%%%%%%%%%%%%%%%%%%%%%%%%%%%%%%%%%%%%%%%%
%%%%% Setting %%%%%%%%%%%%%%%%%%%%%%%%%%
%%%%%%%%%%%%%%%%%%%%%%%%%%%%%%%%%%%%%%%%

%\oddsidemargin=10mm  
%\topmargin=-15mm     
\textwidth=150mm     
\textheight=240mm     


\pagenumbering{arabic}  

%\parindent=0pt 
%\parskip=7pt   
\frenchspacing
%\linespread
\setlength{\abovecaptionskip}{8pt plus 3pt minus 3pt}

%%%%%%%%%%%%%%%%%%   Particles Macros  %%%%%%%%%%%%%%%%%%%%%%%%%%%%%

\newcommand{\my}[1]{{\rm #1}}

\newcommand{\proton}{\my{p}}

\newcommand{\higgs}{\my{H}}
\newcommand{\gluon}{\my{g}}
\newcommand{\pik}{\ensuremath{\pi/\my{K}}}

\newcommand{\photon}{\gamma}
\newcommand{\wboson}{W}
\newcommand{\zboson}{Z}

\newcommand{\electron}{e^-}
\newcommand{\muon}{\mu^-}
\newcommand{\tauon}{\tau^-}

\newcommand{\eneutrino}{\nu_e}
\newcommand{\mneutrino}{\nu_\mu}
\newcommand{\tneutrino}{\nu_\tau}

\newcommand{\qu}{u}
\newcommand{\qd}{d}
\newcommand{\qc}{c}
\newcommand{\qs}{s}
\newcommand{\qt}{t}
\newcommand{\qb}{b}

%%%%%%%%%%%%%%%%%%%%%%%%%%%%%%%%%%%%%%%%%%%%%% Personal Macros %%%%%%%%%%%%%%%%%%%
%%%%%%%%%%%%%%%%%%%%%%%%%%%%%%%%%%%%%%%%% 

%\newcommand{\oo}{\texttt{O$^2$}}
\newcommand{\ROOT}{ROOT}

\newcommand{\e}{\mathrm{e}}
\newcommand{\Lowq}{Low\,-\,\ensuremath{Q^2}~}
\newcommand{\lowq}{low\,-\,\ensuremath{Q^2}~}

\newcommand{\cor}[1]{\textcolor[rgb]{1,0,0}{#1}}
%\def\cor{}
\newcommand{\keep}[1]{\textcolor[rgb]{0,1,0}{#1}}

\newcommand{\clarify}[1]{\textcolor[rgb]{0,0,1}{#1}}

\newcommand{\abs}[1]{\lvert #1 \rvert}
\newcommand{\diff}[1]{\text{d}#1}

\newcommand{\R}{\mathbb{R}}
\newcommand{\sv}[2]{\forall^{\star} #1 \in #2}

\newcommand{\maybegeq}{\overset{?}{\geq}}

\newcommand{\dol}[1]{_{\text{#1}}}
\newcommand{\hor}[1]{^{\text{#1}}}
\newcommand{\bra}[1]{\ensuremath{\langle #1 |}}
\newcommand{\ket}[1]{\ensuremath{| #1 \rangle}}
\newcommand{\braket}[2]{\ensuremath{\left\langle #1 | #2 \right\rangle}}
%\let\theta\vartheta


%%%%%%%%%%%%%%%%%%%%%%%%%%%%%%%%%%%%%%%%%%%%% Document %%%%%%%%%%%%%%%%%%%%%%%%%%%%%%%%%%%%%%%%%%%%%%%%%%%%%%%%%%%%%%%%%%%%
\begin{document}

%%%% 1. strana (Titulni strana) %%%%%%%%

\thispagestyle{empty}

\begin{center}

{\large \bf \cvut} 
    
{\large \bf \fjfi}
    
\vspace{15mm}

{\large \bf \kf} 
    
{\large \bf \obor}
    
\vspace{10mm} 

\epsfysize=30mm 
\epsffile{img/logo_FJFI.pdf}  

\vspace{15mm}
  
{\Large VÝZKUMNÝ ÚKOL}

\vspace{15mm}
   
   
{\Huge \bf \nazevcz} 
 

\vspace{6mm}

   
   \vfill
   {\large
    
    Praha, 2025 \hfill \autor}
\end{center}

%%%%%%%%%%%%%%%%%%%%%%%%%%%%%%%%%%%%%%%%%%%%%%%%%%%%%%%%%%%%%%%%%%%%%%%%%%%%%%%
\newpage
%%%%%%%%%%%%%%%%%%%%%%%%%%%%%%%%%%%%%%%%%%%%%%%%%%%%%%%%%%%%%%%%%%%%%%%%%%%%%%%

\thispagestyle{empty}

\begin{center}

{\large \bf \cvuten} 
    
{\large \bf \fjfien}
    
\vspace{15mm}

{\large \bf \kfen} 
    
{\large \bf \oboren}
    
\vspace{10mm} 

\epsfysize=30mm 
\epsffile{img/logo_FNSPE_cb.pdf}  

\vspace{15mm}
  
{\Large RESEARCH TASK}

\vspace{15mm}
 
{\Huge \bf \nazeven}

\vspace{6mm}

   
   \vfill
   {\large
    
    Prague, 2025 \hfill \autor}
\end{center}



%%%% 2. - 3. strana (Zadání BP) %%%%%%%

%\includepdf[]{Zadani_BP.pdf}
%\includepdf[]{../zadanie.pdf}

%%%% 4. strana (Prohlaseni)
%\includepdf[]{../prohlaseni_nosig.pdf}

%%%% 5. strana (Podekovani)

\newpage
\thispagestyle{empty}

% \begin{center}
%   {\scriptsize Tempora mutantur, nos et mutamur in illis.}
% \end{center}

\vfill 
\noindent {\bf Acknowledgements}
\bigskip
\\
\indent %text
\begin{flushright}
\autor
\end{flushright} 

%%%% 6. strana (Abstrakt CZ)

\newpage
\thispagestyle{empty}

{\bf Bibliografický záznam}

\vspace{5mm} 

\begin{tabular}{p{100pt}l}
  {\em Název práce:} & {\bf \footnotesize\nazevcz} \\
  {\em Autor:} & \autor \\
   &\cvut, \\   
   &\fjfi,\\
   &\kf\\ 
  {\em Studijní program:} & \obor\\ 
  {\em Vedoucí práce:} & \vedouci \\ 
   &\cvut, \\   
   &\fjfi,\\
   &\kf\\
  {\em Konzultant:} & \konzultant \\ 
   &\cvut, \\   
   &\fjfi,\\
   &\kf\\
  {\em Akademický rok:} & \akrok\\
  %{\em Počet stran:} & \pageref{LastPage} \\ %doplniť
  {\em Klíčová slova:} & \klicovaslovacz\\
\end{tabular}

\vspace{10mm}

{\bf Abstrakt}
\vspace{5mm}\\
\indent \abstraktcz


%%TEXT


%%%% 7. strana (Abstrakt EN)

\newpage
\thispagestyle{empty}

{\bf Bibliographic entry}

\vspace{5mm} 

\begin{tabular}{p{100pt}l}
  {\em Title:} & {\bf \footnotesize\nazeven} \\
  {\em Author:} & \autor \\
   &\cvuten, \\   
   &\fjfien,\\
   &\kfen\\ 
  {\em Degree programme:} & \oboren\\ 
  {\em Supervisor:} & \vedouci \\ 
   &\cvuten, \\   
   &\fjfien,\\
   &\kfen\\ 
  {\em Consultant:} & \konzultant \\ 
   &\cvuten, \\   
   &\fjfien,\\
   &\kfen\\ 
  {\em Academic year:} & \akrok\\
  %% {\em Number of pages:} &~\pageref{LastPage} \\ %doplniť
  {\em Keywords:} & \klicovaslovaen\\

\end{tabular}

\vspace{10mm}

{\bf Abstract}
\vspace{5mm}\\
\indent \abstrakten



%%TEXT


%%%%%%%%%%%% Obsah %%%%%%%%%%%%%%%%


\tableofcontents

%\listoffigures

%%%%%%%%%%  Text  %%%%%%%%%

%%%%%%%%%%%%%%%%%%%%%%%%%%%%%%%%%%%%%%%%%%%%% Uvod %%%%%%%%%%%%%%%%%%%%%%%%%%%%%%%%%%%%%%%%%%%%%%%%%%%%%%%%%%%%%%%%%
\chapter*{Introduction}\label{cha:intro} % chktex 24
\addcontentsline{toc}{chapter}{Introduction}
%\vspace{-15pt}

if hello just means goodbye then honey better walk away


% \newpage 
% \
% \newpage

%%%%%%%%%%%%%%%%%%%%%%%%%%%%%%%%%%%%%%%%% kapitola Physics %%%%%%%%%%%%%%%%%%%%%%%%%%%%%%%%%%%%%%%%%%%%%%%%%%%%%%%

% \newpage 
% \
% \newpage

%%%%%%%%%%%%%%%%%%%%%%%%%%%%%%%%%%%%%%%% kapitola ePIC %%%%%%%%%%%%%%%%%%%%%%%%%%%%%%%%%%%%%%%%%%%%%%%%%%%%%%%


% \newpage 
\chapter{ePIC Experiment}\label{cha:epic} % chktex 24
%uviesť tým, že ePIC je pre EIC to, čo STAR pre RHIC
\begin{figure}[H]
    \centering
    \includegraphics[width=.9\linewidth]{img/ePIC_skp.png}
    \caption{Modifed from \url{https://eic.jlab.org/}}
    \label{fig:epic:epic}
\end{figure}

\section{Magnet Systems}

\section{Tracking}
\subsection{Silicon Vertex Tracker}
\subsection{MPGD}
\subsection{AC-LGAD (ToF)}

\section{Particle Identification (PID)}
\subsection{pfRICH}
\subsection{dRICH}
\subsection{hpDIRC}


\section{Electromagnetic Calorimetry}
\subsection{Backward Endcap}
\subsection{Barrel}
\subsection{Forward Endcap}

\section{Hadronic Calorimetry}
\subsection{Backward Endcap}
\subsection{Barrel}
\subsection{Forward Endcap}

\section{Far Backward Detectors}
\subsection{Luminosity detectors}
\subsection{Low-$Q^2$ taggers}

\section{Far Forward Detectors}
\subsection{Roman pots and Off-momentum detectors}
\subsection{Zero-degree calorimeters}


% \newpage

%%%%%%%%%%%%%%%%%%%%%%%%%%%%%%%%%%%%%%%% kapitola nHCal %%%%%%%%%%%%%%%%%%%%%%%%%%%%%%%%%%%%%%%%%%%%%%%%%%%%%%%%%%%%


% \newpage 
\chapter{nHCal}\label{cha:nHCal} % chktex 24
extensively from pre-TDR - new iteration in two weeks - is it worth the wait?

WHERE IS THE GOOGLE DOC? 

Overview from some Leszek's presentation? is Leszek relevant?

\section{Motivation}
still tail catcher of nECal (what is that really, only of that?)

start with HERA (maybe) - then continue from that ("to not make the same mistake")

Vector meson - the matrix image + the 012K plots

only for e + Au and phi, or also e + p, and J/psi?

\section{Construction}
realistic dimensions and location

tiling? is it really important?

does clustering make sense to mention? - probably somewhere else (simulations)

changes?

sampling, N layers, ... ok, but what about material e.g.?

sampling fraction - possible to be compensating (Elke says NO)? what did Subhadip prove, then? - how achieved? how calculated?

but what about true construction? does Leszek now? does anybody?

two images from BP? or something else? cite myself?

anything about neutrons? meaningful?

is tilt usable? if for VU, also for DP?

\section{?}





% \newpage

%%%%%%%%%%%%%%%%%%%%%%%%%%%%%%%%%%%%%%%% kapitola Tools %%%%%%%%%%%%%%%%%%%%%%%%%%%%%%%%%%%%%%%%%%%%%%%%%%%%%%%%%%%%


% \newpage 
% \
% \newpage

%%%%%%%%%%%%%%%%%%%%%%%%%%%%%%%%%%%%%%%% kapitola Simulations %%%%%%%%%%%%%%%%%%%%%%%%%%%%%%%%%%%%%%%%%%%%%%%%%%%%%%%%%


% \newpage 
% \
% \newpage

%%%%%%%%%%%%%%%%%%%%%%%%%%%%%%%%%%%%%%%%%%%%%% Zaver %%%%%%%%%%%%%%%%%%%%%%%%%%%%%%%%%%%%%%%%%%%%%%%%%%%%%%%%%%%%%%

\chapter*{Summary}\label{cha:outro} % chktex 24
\addcontentsline{toc}{chapter}{Summary}

if it's meant to be then it will be




%%%%%%%%% Literatura %%%%%%%%%%%%
\urlstyle{same}
% \thispagestyle{plain}
\begin{thebibliography}{60}
\addcontentsline{toc}{chapter}{Bibliography}

    \bibitem{YR} R. Abdul Khalek et al., "Science Requirements and Detector Concepts for the Electron-Ion Collider: EIC Yellow Report," \textit{Nuclear Physics A}, vol. 1026, a. 122447, October 2021. [Online]. Available: \url{https://doi.org/10.1016/j.nuclphysa.2022.122447}. [Accessed: 20-Dec-2023].

    \bibitem{wiki} ePIC Collaboration, "ePIC Experiment Wiki," \textit{wiki.bnl.gov}, 2024. [Online]. Available: \url{https://wiki.bnl.gov/EPIC/index.php?title=Main_Page}. [Accessed: 27-Feb-2024].

    \bibitem{rhic} Brookhaven National Laboratory, "Relativistic Heavy Ion Collider webpage," \textit{www.bnl.gov}, 2024. [Online]. Available: \url{https://www.bnl.gov/rhic}. [Accessed: 23-Jul-2024].

    \bibitem{whitepaper} A. Accardi et al., "Electron-Ion Collider: The next QCD frontier," \textit{The European Physical Journal A}, vol. 52, a. 268, 2016. [Online]. Available: Springer Link, \url{http://www.springer.com} [Accessed: 20-Dec-2023].

    \bibitem{CDR} U.S. Department of Energy, "Electron Ion Collider Conceptual Design Report 2021," \textit{technical report}, 2021. [Online] Available: \url{https://doi.org/10.x2172/1765663}. [Accessed: 05-Apr-2024].

    \bibitem{grupen_wigmans} R. Wigmans, "Calorimetry," in \textit{Handbook of Particle Detection and Imaging}, C. Grupen and I. Buvat, Eds. Berlin, Heidelberg: Springer, 2012, pp. 497-517. [Online]. Available: \url{https://doi.org/10.1007/978-3-642-13271-1_20}. [Accessed: 12-Apr-2024].

    \bibitem{livan_calorimetry} M. Livan and R. Wigmans, \textit{Calorimetry for Collider Physics, an Introduction}. Cham: Springer, 2019. 

    \bibitem{grupen_particle-detectors} C. Grupen and B. A. Shwartz, \textit{Particle Detectors}. 2nd ed. Cambridge: Cambridge University Press, 2008.

    \bibitem{hanagaki} K. Hanagaki, J. Tanaka, M. Tomoto and Y. Yamazaki, "Particle Identification," in \textit{Experimental Techniques in Modern High-Energy Physics: A Beginner‘s Guide}, K. Hanagaki, J. Tanaka, M. Tomoto and Y. Yamazaki, Eds. Tokyo: Springer. 2022, pp. 69-114. [Online]. Available: \url{ https://doi.org/10.1007/978-4-431-56931-2_6}. [Accessed: 12-Apr-2024].

    \bibitem{lippmann} Ch. Lippmann, "Particle Identification," \textit{Nuclear Instruments and Methods in Physics Research Section A: Accelerators, Spectrometers, Detectors and Associated Equipment}, vol. 666, pp. 148-172, February 2012. [Online]. Available: \url{https://doi.org/10.1016/j.nima.2011.03.009}. [Accessed: 12-Apr-2024].

    \bibitem{dd4hep} M. Frank, F. Gaede, M. Petric and A. Sailer, "DD4hep webpage" \textit{dd4hep.web.cern.ch}, 2023. [Online] Available: \url{https://dd4hep.web.cern.ch/}. [Accessed: 02-May-2024].

    \bibitem{DD4hepManual} M. Frank, F. Gaede, M. Petric and A. Sailer, "DD4hep User Manual," July 24, 2024. [Online] Available: \url{https://dd4hep.web.cern.ch/dd4hep/usermanuals/DD4hepManual/DD4hepManual.pdf}. [Accessed: 02-Jun-2024].

    \bibitem{JANA2} D. Lawrence, A. Boehnlein, N. Brei and D. Romanov, "JANA2: Multithreaded Event Reconstruction," \textit{Journal of Physics: Conference Series}, vol. 1525, a. 012032, 2020. [Online] Available: \url{https://dx.doi.org/10.1088/1742-6596/1525/1/012032}. [Accessed: 02-Jul-2024].

    \bibitem{eicrecon_page} "EICrecon webpage," [Online]. Available: \url{https://eic.github.io/}. [Accessed: 23-May-2024].

    \bibitem{hepmc3} A. Buckley et al., "The HepMC3 event record library for Monte Carlo event generators," \textit{Computer Physics Communications}, vol. 260, a. 107310, March 2021. [Online]. Available: \url{https://doi.org/10.1016/j.cpc.2020.107310}. [Accessed: 02-Jul-2024].

    \bibitem{podio} AIDAsoft, "podio GitHub repository," \textit{github.com}, 2024. [Online]. Available: \url{https://github.com/AIDASoft/podio}. [Accessed: 22-Jul-2024].
     
\end{thebibliography}

% \bibliography{bib_ref.bib}
% \bibliographystyle{unsrt} % nebo něco jinýho, podle toho stylu co je potřeba (stačí googlit)


%%%%%%%%%%  Prilohy  %%%%%%%%%%%%%%%%%%   

%\input{kapitoly/appendix}


\end{document}
