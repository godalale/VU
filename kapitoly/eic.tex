\chapter{Electron-Ion Collider}\label{cha:EIC} % chktex 24

The Electron-Ion Collider (EIC) is a planned accelerator facility to be built at Brookhaven National Laboratory [cite BNL] in the place of today's Relativistic Heavy Ion Collider (known as RHIC). Contrary to RHIC, which was built as an ion-ion collider, EIC will open new possibilities of probing the structure of nucleons by colliding the more complicated ions  with the comparably \enquote{simple-structured} electrons. Its versatile design will allow for the usage of a wide range of these ions: from protons (hydrogen ions) up to ions of uranium [cite Silvia DIS].

The key scientific questions to be addressed by the EIC will be integrated into the final design of the whole accelerator complex, as well as of every experiment and its detectors within. One of these is investigating fundamental properties of nucleons, such as their mass and spin, and discovering how they emerge from the interactions of the partons (gluons and quarks) they are made of. Speaking of the partons, another puzzle for the EIC is to solve how they are distributed inside nucleons, both in momentum and in position. It is also of high interest to discover how they interact with a nuclear medium, what makes the confined hadronic states emerge from these partons, and if the density of gluons keeps growing with increasing energies or if it reaches a point of saturation [cite YR].

In order to solve these conundrums, there are some performance requirements imposed on the design of the EIC. Among them is the large range and variability of center of mass energies, which should range from approximately 20 GeV, up to 100 GeV with a future possibility of reaching 140 GeV. This is needed in order to effectively map out the nuclear structure through various scales of Bjorken $x$. Next, th EIC will boast extraordinarily high luminosity of up to $10^{34}$\,cm$^{-2}$\,sec$^{-1}$ in order to gain access to rare probes and rare events [cite LRP2015]. The peak luminosities for different combinations of electron and proton energy are shown in Figure~\ref{fig:eic:lumi}. Compared to its spiritual predecessor (that is the collider HERA\footnote{HERA (or \emph{Hadron-Elektron Ring Anlage} in German) was an electron-proton collider in the DESY laboratory in Hamburg, Germany. It operated between the years 1997 and 2007 [cite HERA].}), which reached a peak luminosity of $5 \cdot 10^{-31}$\,cm$^{-2}$\,sec$^{-1}$~[cite HERA], the EIC is designed to achieve luminosity roughly three orders of magnitude greater. Lastly, there needs to be a open possibility for a second (complementary) detector, which would provide an independent confirmation for discovery measurements and an overall expansion of scientific opportunities [cite source2].

\section{From RHIC to EIC}
For more than twenty years RHIC has been a cornerstone of nuclear and particle physics. Commissioned in 2000, it was the first collider able to collide ions heavier than protons, creating the hot, dense conditions akin to those of the early universe [cite RHIC-facts]. It was built as the result of a compromise between particle physicists interested in the mechanism of multi-particle production in high-energy hadron collisions and nuclear physicists interested in investigating the nuclear equation of state with the highly compressed nuclear matter. RHIC also pioneered the use of spin-polarized\footnote{Particles are said to be \emph{polarized} if their spins have a preferential orientation so that there exists a direction, for which two possible spin states are not equally populated [cite polarized-electrons].} proton beams [cite RHIC-program].

The legacy of innovation will continue with the EIC, which will also incorporate a polarized electron beam into its design. The EIC, shown in Figure~\ref{fig:eic:eic}, will reuse many elements of the existing RHIC infrastructure, some of which are: the tunnel, the \enquote{Yellow} Ring (shown in Figure~\ref{fig:eic:rhic}), a small section of the \enquote{Blue} Ring, cryogenic systems, some of the magnets, and more. The main changes will involve adding a new electron accelerator and modifying the existing ion accelerator to accommodate electron-ion collisions [cite CDR]. This approach will significantly reduce costs, while it will also build upon the proven capabilities of RHIC.

\begin{figure}[ht]
    \centering
    \begin{subfigure}{.25\linewidth}
        \includegraphics[width=\linewidth]{img/rhic.jpg}
        \caption{}
        \label{fig:eic:rhic}
    \end{subfigure}\hspace{.2\linewidth}
    \begin{subfigure}{.35\linewidth}
        \includegraphics[width=\linewidth]{img/IR.pdf}
        \vspace{.5cm}
        \caption{}
        \label{fig:eic:IR}
    \end{subfigure}
    \caption{(a) Layout of the RHIC facility. The \enquote{Yellow} Ring planned for reuse in the EIC is shown in yellow. Taken from [cite RHIC image]. (b) PLACEHOLDER Schematic of the Interaction Regions (IR) of RHIC, consistent with the EIC.}
\end{figure}

\begin{figure}[ht]
    \centering
    \includegraphics[width=.85\linewidth]{img/EIC_new.png}
    \caption{Current design of the EIC. Taken from [cite EIC-new-image].}
    \label{fig:eic:eic}
\end{figure}

\begin{figure}[ht]
    \centering
    \includegraphics[width=.85\linewidth]{img/luminosity2_placeholder.pdf}
    \caption{PLACEHOLDER Peak luminosities for different electron$\times$ion energy combinations. Taken from [cite MOA1].} % aj v eic_cdr_final.pdf na s. 18
    \label{fig:eic:lumi}
\end{figure}

\section{Hadrons}
The hadronic component of the EIC will greatly benefit from the reused infrastructure of RHIC. Recalling again the connection with HERA, which was limited to electron-proton collisions, the EIC will accommodate a wide variety of heavier ions. These hadron beams, at least those made of protons and light nuclei (such as \isotope[3]{He}), will be highly polarized, reaching up to 70\% [cite LRP2015].

\subsection{Particle sources}
Following in the steps of RHIC, the EIC will also utilize a polarized hadron beam. This lends an opportunity to reuse the currently existing infrastructure, especially for creating and accelerating polarized protons. 

Production of negatively charged hydrogen ions in RHIC is handled by the Optically Pumped Polarized Ion Source (OPPIS), which has been in use since the beginning of RHIC's program and during its lifetime has gone through several upgrades. [cite polA-sources] Polarization in the OPPIS is achieved with a circularly polarized laser beam, which is used to polarize an atom of rubidium. Meanwhile a hydrogen atom is ionized by an atom of helium. The now ionized hydrogen atom (that is, a proton) captures a polarized electron from the atom of rubidium. The polarization is then transferred from the electron to the proton. At the end of this process, the polarized but electrically neutral hydrogen atom has to capture a second electron, so it may be then accelerated. [cite polP-EIC]

In order to fulfill the scientific plan, there is a strong need and demand for a beam of polarized neutrons. Unfortunately, as it is evident, neutrons by themselves cannot be accelerated. It is necessary to resort to some charged ion that would contain the desired neutrons. The first logical step from a lone neutron would be some ionized form of deuterium, which contains an additional proton. A feasibility study has already been done for acceleration of polarized deuterons in the EIC, which deemed it possible [cite deuteron-feasibility]. However, the small magnetic moment of deuterium could pose a challenge for spin manipulation. That makes helions (\isotope[3]{He}) a preferred alternative, as the most abundant state of \isotope[3]{He} has the protons' spins anti-aligned, which makes the neutron practically carry the nuclear spin. Therefore, helions make for an effective spin-polarized neutron source [cite EIC-scientific-beams]. They are to be produced in the Electron Beam Ion Source (EBIS), which will also remain as the particle source of choice for heavier ions for the EIC [cite optically-pumped]. Possibility of extending the polarization to heavier ions, such as \isotope[6]{Li} or \isotope[7]{Li}, is still under discussion, as it could prove beneficial for gaining new insights into quark-gluon dynamics within the nuclear medium, or even beyond [cite scientific-beams]. 

\subsection{Pre-acceleration}
Before entering the EIC's Hadron Storage Ring (HSR), the beam needs to undergo several steps of preacceleration. This infrastructure will be inherited from the initial acceleration systems used by RHIC. Travelling through it, protons coming from a source get first accelerated in a Linear Accelerator (LINAC) to energy of 200~MeV, then pass through a booster synchrotron (known as Booster) with energy of 1.5~GeV, and finally they enter the Alternating Gradient Synchrotron (AGS), from which they come out with energy of 25~GeV [cite Nagaitsev Frascati]. All of the aforementioned accelerators are shown in Figure~\ref{fig:eic:eic}.

\subsection{Hadron Storage Ring}
The reuse of the arcs of at least one of RHIC's storage rings has been a part of the design plan of the EIC since the beginning [cite CDR]. Ultimately, after assessing the tunnel and considering the requirements for accessibility, the “Yellow” ring was chosen to be kept in its entirety. Its existing arcs will live on as the HSR after removal of a number of magnets from Interaction Region 6 (IR6) and from Interaction Region 2 (IR2), due to significant modifications in these regions [cite EIC-design-highlights]. Positions of these are shown schematically in Figure~\ref{fig:eic:IR}. Because the EIC will use a much larger number of bunches of shorter length than RHIC, resulting in the beam current being increased by a factor of 3 [cite RHIC-to-EIC-HSR], new actively cooled beam screens will be inserted into the superconducting magnet beam pipes. These sleeves will be coated with copper to reduce electric resistivity and with a secondary layer of amorphous carbon to prevent the accumulation of electron clouds [cite scientific-beams]. 

In order to preserve polarization of the beam, there are already two Siberian snakes\footnote{Siberian snake is a device that rotates the spin vectors of the beam particles by \ang{180} each turn, in order to mitigate the effect of depolarizing resonances in circular accelerators [cite siberian-snakes].} present in RHIC. Additional four will be installed in the HSR - two of those coming directly from the RHIC's \enquote{Blue} ring, and two will be built from spin rotators from the same ring [cite scientific-beams].

The only section of RHIC's \enquote{Blue} ring to be fully reused by the EIC will serve as a low-energy bypass for hadrons. This is necessary because unlike relativistic electrons, whose speed remains nearly constant with energy, the velocity of low-energy ions varies significantly from those of higher energies. The bypass, which makes the circumference of the whole HSR smaller, enables synchronization of the two beams [cite CDR]. The relevant sector of the EIC with and without the low-energy bypass is shown in Figure~\ref{fig:eic:tunnel}, where the colors of the used rings correspond to their names in RHIC nomenclature.

\begin{figure}[ht]
    \centering
    \includegraphics[width=.9\linewidth]{img/tunnel.png}
    \caption{PLACEHOLDER Sector 1 of the EIC tunnel with and without the low-energy bypass.}
    \label{fig:eic:tunnel}
\end{figure}

\section{Electrons}
In keeping with the hadron beam, also the energy of the electron beam will be variable. The chosen beam energies are 5 GeV, 10 GeV, and 18 GeV. Consistent with the hadron beam, the electron beam must reach the same high levels of average polarization, that is 70\% [cite LRP2015]. 

\subsection{Particle source}
The electron beam in the EIC will be polarized right from the source [cite polarized-electron-beams]. That is in contrast with HERA, where the electron beam polarization was achieved solely by the Sokolov-Ternov effect\footnote{Unpolarized electrons injected into a storage ring tend to become polarized over a period of time as they align their spin anti-parallel to the magnetic field, while emitting synchrotron radiation [cite sokolov-ternov].} [cite HERA]. The electrons will be produced by an inverted high voltage direct current (HVDC) gun [cite dc-gun], using a green (532 nm) laser on a multi-alkali CsK$_2$Sb photocathode [cite electron-source-laser]. 

\subsection{Pre-acceleration}
The electrons coming from the photocathode source will pass through a normal-conducting LINAC, in which they will be accelerated to 750 MeV. The Beam Accumulator Ring (BAR) will then be used to provide 7 nC or 28 nC bunches by stacking multiple low-charge pulses. These now high-charge bunches will pass to the Rapid Cycling Synchrotron (RCS), which will ramp the electron beam up to energies of several GeV, making them suitable for injection to the Electron Storage Ring (ESR) [cite linac-to-bar-rcs, EIC-scientific-beams].

\subsection{Electron Storage Ring}
Unlike the HSR, for which the existing infrastructure of RHIC may be reused, the ESR will have to be built from scratch, albeit in the same tunnel. As mentioned previously, the injected electron bunches will enter the ESR already with the desired spin orientation and at full intensity and energy. Depending on the spin pattern (spin \enquote{up} or \enquote{down}) the polarization will decay due to the Sokolov-Ternov effect at different rates. That means that bunches with their spins parallel to the dipole filed will depolarize at higher rate and will need to be replaced more frequently than those with their spins antiparellel to the field. This will ensure the same average polarization [cite scientific-beams].

\section{Interaction Region 6 (IR6)}
Current design plans for the EIC rely on one IR, located at the 6 o'clock position, replacing the Solenoidal Tracker At RHIC (STAR) Experiment. It is marked in Figure~\ref{fig:eic:eic} by the logo of the ePIC Experiment. Detailed description of the detector subsystems that comprise it will be a subject of a dedicated chapter in this text. The design plans of the EIC allow for the possibility of a second interaction region, located at the 8 o'clock position, which would stand in the place of RHIC's sPHENIX Experiment.

The detector to be built in the IR6 will by itself occupy some space, approximately 5 meters, around the point of the collision. This limits the closest possible distance at which the last focusing elements may be located. To accommodate this geometry for both electron and hadron beams, a crossing angle of 25 milliradians will be introduced. This configuration enables the installation of the final focusing quadrupoles for each beam side-by-side while eliminating the need for separator dipoles, which would otherwise generate unwanted synchrotron radiation in the detector region. The luminosity loss caused by the crossing angle is compensated by a system of crab cavities located symmetrically on both sides of the IP, which rotate the bunches so that they effectively collide head-on with the highest possible overlap [cite CDR]. The rotation caused by the crab cavities and the resulting change of the overlap is shown in Figure~\ref{fig:eic:crab}.

\begin{figure}[ht]
    \centering
    \includegraphics[width=.6\linewidth]{img/crab_cavities.jpg}
    \caption{Crab cavities used to rotate the particle bunches before the collision to maximize the overlap of the colliding bunches. Taken from [cite crab-cavities].}
    \label{fig:eic:crab}
\end{figure}

